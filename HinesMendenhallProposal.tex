\documentclass[12pt]{article}

\usepackage{amssymb,amsmath}
\usepackage[margin=1.0in]{geometry}
\usepackage{fancyhdr} % required for custom header
\usepackage{graphicx}
\usepackage{listings}
\usepackage{courier}
\usepackage{natbib}
\usepackage[usenames,dvipsnames]{color}

%set up the header
\pagestyle{fancy}
\lhead{Trever T. Hines}
\chead{RO 16-15 Reasearch Proposal}
\rhead{\today}

\setlength{\headheight}{15pt}
\renewcommand\headrulewidth{1.0pt} % Size of the header rule

%% Title
%%------------------------------------------------------------------------------
\title{	
 \rule{\headwidth}{1.0pt}
 Research proposal for RO 16-15:
 Novel crustal deformation models \
 characterizing earthquake hazard and its uncertainties in the western U.S.\
 \rule{\headwidth}{1.0pt}
 \author{Trever T. Hines}}

\begin{document}
\maketitle

\section*{Research Objective}
Finite fault rupture models are a 

In order to understand Quantifying seismic hazard requires an understanding of the faults have slipped in the past.  it is important to know the 
C
Quantifying the uncertainty The uncertainty 
\citep{Hines2016} and \citet{Hines2016}
\section*{Work Plan}

\section*{Expenses}

%\bibliographystyle{mlalike}
\bibliography{mybib}
\end{document}