\documentclass[12pt]{article}

\usepackage{amssymb,amsmath}
\usepackage[margin=1.0in]{geometry}
\usepackage{fancyhdr} % required for custom header
\usepackage{graphicx}
\usepackage{listings}
\usepackage{courier}
\usepackage{natbib}
\usepackage[usenames,dvipsnames]{color}

%set up the header
\pagestyle{fancy}
\lhead{Trever T. Hines}
\chead{RO 16-15 Reasearch Proposal}
\rhead{\today}

\setlength{\headheight}{15pt}
\renewcommand\headrulewidth{1.0pt} % Size of the header rule

%% Title
%%------------------------------------------------------------------------------
\title{	
 \rule{\headwidth}{1.0pt}
 Research proposal for RO 16-15:
 Novel crustal deformation models \
 characterizing earthquake hazard and its uncertainties in the western U.S.\
 \rule{\headwidth}{1.0pt}
 \author{Trever T. Hines}}

\begin{document}
\maketitle

\section*{Research Objective}
An accurate quantification of seismic hazard requires estimates of where faults have previously ruptured as well as the uncertainties on those estimates.  This research proposal aims to address the latter. Finite fault rupture models, which are obtained through a regularized least-squares inversion of geodetic data, lacks meaningful uncertainty quantifications.  The reason being that model uncertainties are a function of the subjectively chosen regularization.  However, regularization is a modelling necessity because fault slip inversions are an otherwise ill-posed problem.  The task is then to find a means of regularizing finite fault rupture models that has physical motivation and is well quantified.  In such case, regularization can be viewed as adding a prior constraint on the slip model in a Bayesian sense.  There are three empirically derived relationships which we draw our prior constraints from: 1) faults slip in the direction of maximum shear stress 2) slip occurs when the Mohr-Coulomb failure criteria is exceeded 3) shear stress drop on any earthquake, regardless of rupture dimensions, tends to be in the range of 0.1 to 10 MPa.  The first two relationships would require an understanding of ambient stresses in order to use then as a prior constraint for fault slip inversions.  The are also complications with the third relationship because it assumes that earthquakes produce a uniform stress drop on an geometrically simple fault.  Stress drop is an average over the extent of a fault rupture. Since stress drop is invariant to the size of the rupture, we postulate that the stress drop at each point on the fault should also be roughly subject to the same bounds of 0.1 to 10.0 MPa.  Since we do not know the spatial extent of a rupture, we cannot bound stress drop on each fault patch to be between 0.1 to 10.0 MPa since this would cause the entire fault segment to rupture.  The beyond the  coseismic rupture, stresses will increase and then taper to zero further out.  We can then say that the stress change everywhere on the fault should be greater than 10 MPa. will have a stress increase and   


we then posit that Stress drop is an average over the extent of a fault rupture.  It is then not clear how a stress drop can be used in a fault slip model, where the extent of the rupture is unknown.  However, since stress drop is invariant to 

   It is not immediately apparent how bounds on stress drop can translate to constraints for the more geometrically complicated finite fault model.           
   

are often presented as a single which is sorely lacking in geodetically derived finite fault rupture models.

has previously ruptured and   
Finite fault rupture models are a 

In order to understand Quantifying seismic hazard requires an understanding of the faults have slipped in the past.  it is important to know the 
C
Quantifying the uncertainty The uncertainty 
\citep{Hines2016} and \citet{Hines2016}
\section*{Work Plan}

\section*{Expenses}

%\bibliographystyle{mlalike}
\bibliography{mybib}
\end{document}